\documentclass[]{article}
\begin{document}

\title{Documentation of final submission for assingnment AC circuit solver}
\author{Ayush Singh Pal(2016csj0018),
kulvendra singh(2016csj0026)
}
\maketitle

\begin{abstract}
This document represents the objectives that we were able to achieve which were set during the assignment announcement.
It also give an insight of detailed instructions that are necessary to run the program.
\end{abstract}


\section{Name of all files that are being used}

-assgn2.cpp

-draw.h

-solve.cpp
 
\section{Objectives completed}

-program successfully parse the input netlist format and extract the information like magnnitude and unit.

-It displays the circuit with all components connected according to the data given in netlist format.

-It solve the circuit and prints the results in a text file.

\section{objectives omitted/changes}

omitted objective

-pop out the information when being clicked upon the svg image

-have covered most of the cases but not all  while solving the ac circuit

Changes


-there is a little change in netlist format while entering the input there should be a space between magnitude and unit i.e 10 mH is acceptable instead of 10mH,for part 1 input is accepted according to netlist format

\section{Additional feature}

-The final output file contains a button at top left which allows the user to toggle page in all four direction i.e up,down,left,right.

- it also contain a button to zoom in or out.


\section{Content from read me}

For part 2 go through these given assumptions and guidelines

-The attached file is program to draw an svg image and solve an ac circuit;

-For the solving part we made certain assumption and guidelines which 
are briefly mentioned below;

We have solved the ac circuit in which all elements are connected in series or in parallel with any number of voltage source and  current source;

-there can any number of elements hence we have covered all possible 7 cases which are 

-purely resistive

-purely capacitive

-purely inductive

-lc circuit

-lr circuit

-rc circuit

-lcr circuit

-User are required to follow following guidelines while entering input

-There should be space between magnitude and their unit likewise 10 k is correct while 10k is not valid.

-To enter a value with decimal point u neeed to use unit to express that floating point integer i.e to enter 12.35 u need enter 1235 c where 'c'
stands for centi and is equal to 1/100.so input requires integer unit ,choose your unit wisely so that it can convert your input into floating point integer.

-Following are the unit provided by program with their abbrevation :-

'M'  for mega=1000000;

'k'  for kilo=1000;

'da'  for deca=10;

'd'  for deci=1/10;

'c'  for centi=1/100;

'm'  for milli=1/1000;

'u' for micro=1/1000000;

'n' for nano=1/1000000000;


-For voltage the amplitude only excepts integer with no unit .

-Four testcases are also provided which cover following 4 case;

1)Multiple voltage source with multiple elements connected in sereies

2)Multiple voltage source with multiple elements connected in parallel

3)Multiple current source with multiple elements connected in sereies

4)Multiple current source with multiple elements connected in parallel

-Intially program needs to know  type of circuit out of the above four from user.

-Make sure to follow above guidelines while entering input any violation will result in no result  




	


\end{document}
